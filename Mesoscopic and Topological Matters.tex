\documentclass[12pt,openany]{book}
\usepackage[utf8]{inputenc}
\usepackage[letterpaper,top=2cm,bottom=2cm,left=2cm,right=2cm,marginparwidth=1.75cm]{geometry}
\usepackage{wrapfig}
\usepackage[psdextra, colorlinks=true, allcolors=black]{hyperref}
\usepackage[italian]{babel}
\usepackage{epigraph}
\usepackage{afterpage}
\newcommand\blankpage{%
    \null
    \thispagestyle{empty}%
    \newpage} %serve a lasciare una pagina vuota
\usepackage{import}
\usepackage{physics}
\usepackage[version=4]{mhchem}
\usepackage{amsfonts}
\usepackage{mathtools}
\usepackage{graphicx}
\usepackage{amssymb}
\usepackage{amsmath}
\usepackage{physics}
\usepackage{enumitem}
\usepackage{array}
\usepackage{tikz,pgf}
\usetikzlibrary{snakes}
\usetikzlibrary{shapes}
\usepackage{lipsum}

%servono a mettere i simboli greci nei titoli
\usepackage[open]{bookmark}
\ProvidesFile{puenc-greek.def}
\usepackage{textgreek}

\usepackage{float}
\setlength\parindent{0pt}%e si gode, toglie lo spostmento a destra di una nuova riga
\setlength{\epigraphwidth}{0.5\textwidth}
\usepackage{caption}
\usepackage{subcaption}
\usepackage{fancyhdr}

\newcommand{\comment}[1]{}

\usepackage{ambienti} %pacchetto dove sono definiti i vari ambienti esercizio, approfondimento ecc

%simboli
\newcommand{\E}{È 
\hspace{0.1mm}}

\newcommand{\A}{Å 
\hspace{0.1mm}}

\begin{document}

\thispagestyle{empty}
\begin{center}

\begin{minipage}[c]{0.45\textwidth}
\begin{flushleft}
\includegraphics[width=0.8\textwidth]{logo-unict-orizzontale-grigio.png}
\end{flushleft}
\end{minipage}
\hfill
\begin{minipage}[c]{0.45\textwidth}
\begin{flushright}
\includegraphics[width=\textwidth]{logo_dfa_orizzontale}
\end{flushright}
\end{minipage}\\
\medskip
\hbox to \textwidth{\hrulefill}

\vfill
\vfill

\uppercase{\sc{ \Large{\textbf{Mesoscopic and Topological Matters}}}}\\

\vfill
\large{A cura di Joey Butchers}

\vfill
\vfill
\hbox to \textwidth{\hrulefill}
{\sc Anno Accademico 2024-2025}
\end{center}

\afterpage{\blankpage}
\newpage

\clearpage                       % Otherwise \pagestyle affects the previous page.
{                                % Enclosed in braces so that re-definition is temporary.
  \pagestyle{empty}              % Removes numbers from middle pages.
  \fancypagestyle{plain}         % Re-definition removes numbers from first page.
  {
    \fancyhf{}%                       % Clear all header and footer fields.
    \renewcommand{\headrulewidth}{0pt}% Clear rules (remove these two lines if not desired).
    \renewcommand{\footrulewidth}{0pt}%
  }
  \tableofcontents
  \thispagestyle{empty}          % Removes numbers from last page.
} %roba per mettere l'indice senza numero di pagina ne marks

\newpage

\pagestyle{fancy}
\fancyhf{}
\fancyhead[LE]{\nouppercase{\textbf{\thepage}\hfill\leftmark}}
\fancyhead[RO]{\nouppercase{\rightmark\hfill \textbf{\thepage}}}
\fancypagestyle{plain}{%
\fancyhf{} % cancella tutti i campi di intestazione e pi\‘e di pagina
%\fancyfoot[C]{\bfseries \thepage} % tranne il centro
\renewcommand{\headrulewidth}{0pt}
}

\chapter{Lezione 1}
\import{./Lezioni}{Lezione1}

\end{document}
